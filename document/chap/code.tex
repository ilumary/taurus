\chapter{Code}

\section{Pseudocode}

\begin{algorithm}[H]
  \KwIn{A[0..N-1], value}
  \KwOut{value if found, otherwise not\_found}
  \tcp{Lokale Variablen}
  let \textit{low} = 0
  let \textit{high} = N - 1
  
  \tcp{Es gibt auch noch weitere Loops wie ForEach}
  \While{low \textless= high}
  {
    \tcp{Invariante: value \textgreater{} A[i] for all i \textless low}
    \tcp{Invariante: value \textless{} A[i] for all i \textgreater high}
    mid = (low + high) / 2\;
    \eIf{A[mid] \textgreater value}
    {
      high = mid - 1\;
    }
    {
      \eIf{A[mid] \textless value}
      {
	low = mid + 1\;
      }
      {
	return mid\;
      }
    }
    return not\_found\;
  }
  \label{binary_search}
  \caption{Binärsuche in Pseudocode}
\end{algorithm}

\clearpage

\section{Codeschnipsel}

\begin{codeblock}{hello.rs}{Rust}
  \begin{rustcode}
    fn main() {
      println!("Hello world!");
    }
  \end{rustcode}
\end{codeblock}

\section{Terminalausgabe}

\begin{terminalblock}
  \begin{textcode}
    [user@computer thesis]$ ls -la
    total 12
    drwxr-xr-x  3 user user 4096 Apr 28 19:09 .
    drwxr-xr-x 23 user user 4096 Apr 28 19:09 ..
    drwxr-xr-x  7 user user 4096 Apr 28 20:12 folder$
  \end{textcode} 
\end{terminalblock}








% custom style (c code style)
% \lstdefinestyle{customc}{
%   belowcaptionskip=1\baselineskip,
%   breaklines=true,
%   frame=L,
%   xleftmargin=\parindent,
%   language=C,
%   showstringspaces=false,
%   basicstyle=\footnotesize\ttfamily,
%   keywordstyle=\bfseries\color{green!40!black},
%   commentstyle=\itshape\color{purple!40!black},
%   identifierstyle=\color{blue},
%   stringstyle=\color{orange},
% }
% 
% \lstdefinestyle{customasm}{
%   belowcaptionskip=1\baselineskip,
%   frame=L,
%   xleftmargin=\parindent,
%   language=[x86masm]Assembler,
%   basicstyle=\footnotesize\ttfamily,
%   commentstyle=\itshape\color{purple!40!black},
% }
% 
% \lstset{escapechar=@,style=customc}