\chapter{Introduction}

Over the past four decades, the reliability of the Transmission Control Protocol (TCP) has been the cornerstone
of typical, day-to-day user traffic on the Internet. TCP, renowned for its ability to provide reliable, ordered,
and error-checked data transmission between two applications, has been the foundation in sustaining the
connectivity that underpins our digital experiences. However, the ever-accelerating evolution of network
landscapes, coupled with an intensifying demand for heightened flexibility, security, and superior performance,
has catalyzed in the exploration of alternatives. In response to these demands, the QUIC protocol has emerged
as a solution to next-generation networking, with its roots tracing back to the laboratories of Google.
In a contemporary environment where the static nature of traffic from predominantly home networks has given way
to a dynamic mix of devices, networks and technologies, QUIC represents the much needed next step in networking.
Unlike its predecessor, QUIC is designed to meet the challenges of a modern digital ecosystem, where seamless
and secure connections are imperative across a diverse array of devices and usage scenarios. As our lives become
increasingly interconnected and reliant on a myriad of digital interfaces, QUIC stands as the widely adopted next
step, ensuring not only reliability but also security.

\section{Motivation}

The current TCP/IP stack faces limitations in latency and connection establishment times due to its three-way
handshake mechanism. QUIC, built on top of UDP, offers significant improvements ocer TCP. Firstly, it utilizes a single
handshake for establishing a secure connection, leading to faster connection establishment. Secondly,
QUIC integrates a high performance stream control system combined with a robust error detection system, allowing
for reliable data transmission even with packet loss, improving application performance.

QUIC provides an excellent opportunity to utilise a systems programming language for an implementation in order to
maximise performance while managing resources at byte level. The programming language Rust has a number of benefits
as it is currently gaining traction due to its focus on memory safety, performance, and concurrency. Its ownership
system eliminates dangling pointers and memory leaks, crucial for low-level network programming. Additionally,
Rust's ability to express concurrency efficiently makes it well-suited for handling the asynchronous nature of
network communication.
%and therefore highly beneficial when as utilised by the QUIC protocol.

This thesis aims to develop a basic QUIC implementation in Rust that demonstrates basic connection
handling, data transfer, and the library design. While not a full-fledged production-ready
implementation, this prototype will provide valuable insights into QUICs packet and header designs,
its handshake mechanisms, its security measures and its performace benefits over TCP.

\section{Objective}

This bachelor's thesis objective is to implement a standalone version of the QUIC transport protocol in Rust,
utilizing version rustc 1.72.0 and the UDP-Socket from the Rust standard library as its foundation. The
implementation aims to comprise several features crucial for a functional QUIC server endpoint. These include
the ability to send and receive QUIC packets, parsing, and encryption/decryption of both header and payload data.
Additionally, the system aims to support a fully executed 1-RTT QUIC handshake between two endpoints, employing
TLS 1.3 through the Rustls crate.

In developing this implementation, reference is exclusively made to RFCs 8999-9001, chosen due to a lack of
comparable open-source implementations in Rust. A focus is set on handling all frames within the payload,
with special attention given to Cryptoframes and Dataframes.

The implementation makes use of existing tools, such as the Rustls crate with the "quic" feature for a
complete TLS 1.3 implementation and the Octets crate, serving as a Rust zero-copy mutable Byte Buffer to
facilitate parsing.

Certain features are deliberately omitted due to time constraints. These include 0-RTT packets, version
negotiation between different QUIC versions, connection migration, a client endpoint, asynchronous processing,
and advanced flow control.

The primary communication goal is to establish seamless communication with a Python-implemented client.
This client initiates the handshake and subsequently sends individual data packets as payloads, awaiting
acknowledgment from the Rust-based server. Notably, the Python QUIC client is implemented using the open-source
implementation aioquic.

Furthermore this bachelor's thesis aims to provide a comprehensive exploration of the QUIC protocol, exploring
its intricacies and explaining its functionalities. By examining QUIC's features, particularly its transport,
encryption, multiplexing, and error-handling mechanisms, one can better discern how QUIC stands out in terms
of resilience, efficiency, and security when compared to TCP.

\section{Overview}

Starting with the subsequent chapter, "Fundamentals", the thesis establishes a foundation for understanding
the principles of QUIC. Key terms and definitions are clarified, followed by an introduction into networking
protocol fundamentals. Furthermore, both TCP and UDP are introduced in more detail.

Chapter three forms the core of the thesis, in which QUIC is examined in detail. It starts by exploring QUIC's
historical background, clarifying its goals in addressing problems faced by TCP and other protocols, and continues
with a detailed breakdown of its architecture. Moving into the more practical part of QUIC, the following
subsections break down its connections and the ways in which QUIC deals with different errors and how it recovers
from them. Finally, we examine the security features integrated into QUIC.

Moving into the implementation chapter, the rationale behind choosing Rust as the implementation
language is explained and the scope of the QUIC implementation is set, followed by the development environment specifics
and all external libraries the implementation makes use of. The main part of chapter four deals with the library layout
and dives deep into specifics of connection establishment, packet handling and stream management. Lastly both protocol
and implementation error handling mechanisms will be examined before concluding the chapter with an overview
of the strategies used to test and validate parts of the implementation.

The "Evaluation" chapter critically assesses the implemented QUIC protocol. It addresses encountered challenges,
acknowledges inherent limitations, evaluates performance considerations, provides considerations for future
improvements and modifications, and conducts a comparative analysis with other existing implementations.

In the concluding chapter, key findings are laid out and a conclusion is drawn. The end of this thesis provides
an outlook into the future of QUIC.