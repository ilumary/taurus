\chapter{Evaluation}

\todo{Execution, Auswertung usw}

\section{Challenges}
The vast scope of QUIC inherently leads to high complexity. This is evident from the very beginning - 
the implementation overhead needed to just be able to build an answer to an initial packet is immense.
Throughout the whole implementation process a carefully designed library is crucial to avoid major refactoring and
rewrites later on.

Unfortunately, the relatively new nature of QUIC means there's a scarcity of widely available implementations, tutorials,
and explanations.  This lack of resources makes it more challenging to understand the protocol itself and derive a design
for the QUIC library.

To make matters even more intricate, comprehending existing open-source QUIC implementations becomes an additional hurdle due
to the protocol's vast scope and the inherent time limitations of this theis. The \inlinecode{quinn} library alone has over
32 thousand lines of rust code.

Most of the major setbacks encountered during the implementation process stemmed from the Rustls library. Despite its full support for 
TLS 1.3 and a specifically for QUIC designed API, a significant portion of its functionality are undocumented. This lack of documentation
extends over most of the API, in that the explanations of function parameters is often incomplete or missing and examples with
broader contexts are missing in their entirety.
For instance, when initializing the RustlsConnection, it was unclear that the third parameter (\inlinecode{params}) referred to
RFC section 18. Similarly, when calling \inlinecode{read\_hs()}, which takes a reference to data, it wasn't clear where the
pointer should start - at the beginning of the CRYPTO frame or the TLS message within it. Further testing revealed that
the function returned an error, but not one of the standardized TLS 1.3 errors. To identify the issue, one has to explicitly
poll for TLS alerts using the \inlinecode{alerts()} function which wasn't mentioned anywhere (Fig. \ref{error_handling_read_hs}).

\begin{codeblock}{lib.rs}{Rust}
    \begin{rustcode}
        match self.tls_session.read_hs(crypto_frame.data()) {
            Ok(()) => println!("Successfully read handshake data"),
            Err(err) => {
                eprintln!("Error reading crypto data: {}", err);
                eprintln!("{:?}", self.tls_session.alert().unwrap());
            }
        }
    \end{rustcode}
    \label{error_handling_read_hs}
\end{codeblock}

The biggest challenge, however, was related to the application layer protocol negotiation. This is an obligatory field, carried
in any TLS client hello, and its contents are arbitrary but a standardized list of protocol codes is maintained by IANA, because
when negotiating a protocol, for example http/3, the protocol code has to match on both endpoints. The initial test client, built
with \inlinecode{aioquic} in Python, did not include this field, leading to an error and a TLS alert. After switching the test
client to the example HTTP/3 client provided by the quinn library, the ALPN field was included, but the connection still failed.
The Rustls documentation made no mention of its inability to handle the standardized IANA list. After digging deep
within the library code, it became apparent that ALPN protocols have to be specified manually outside of the usual configuration.
This involved directly setting the field within a specific struct of the RustlsConnection class. 

\section{Future Improvements}

While the current QUIC library provides a solid foundation, it lacks several key features that need to be implemented first, in
order to facilitate QUIC connections. The initial library was designed with the whole protocol in mind and should be
easily expandable without requiring a major restructuring effort. That being said, there are still a number of areas within the
existing codebase that could benefit from optimization and improvement, even if the pure QUIC implementation is feature complete.

\subsection{Asychronous Design}

The current design of the QUIC library operates within a single thread. However, network programming inherently benefits from a
concurrent approach. Rust offers two primary models for achieving concurrency: OS threads and asynchronous programming with a runtime.

Asynchronous programming excels in scenarios dominated by I/O bound tasks, which is precisely the case for network applications
like servers and databases. It significantly reduces CPU and memory overhead compared to traditional thread-based approaches.
An async runtime manages a limited pool of expensive OS threads efficiently. These threads handle a much larger
number of lightweight tasks, enabling significantly more concurrent operations compared to using raw OS threads directly.

Unfortunately, transitioning the current library to a concurrent model would necessitate a major rework, particularly for the
\inlinecode{Connection} and \inlinecode{Endpoint} components. Facilitating concurrent reading and writing of serveral resources 
would require significant changes. For instance, the connection database design might need to be guarded by a mutex (a synchronization
primitive) or even removed entirely in favor of a different approach (see Sec. \ref{connection_db_rework}). Additionally, the
\inlinecode{recv()} function for example would need to return a \inlinecode{Future} object, requiring the use of await
statements to retrieve data in an asynchronous environment.

The transition to an asychronous design would have major performance benefits but would require a major effort as Rusts asynchronous
features are widely known to be very difficult to work with and are still maturing\footnote{\url{https://bitbashing.io/async-rust.html}}.

\subsection{Connection Database Rework in Endpoint} \label{connection_db_rework}

% - current design keeps all the connections in a vector inside the endpoint and uses their indices as handles in a hashmap where the
% current connection id is the key
% - requires lookup for every received packet and every event 
% - possible performance bottleneck if server scales to hundreds of connections with a concurrent programming model
% - a solution would be to give the user the resposibility over hte individual connections
% - Connection and Enpoint structs would be wrapped in structs which expose the direct api of both objects
% - a wrapper for the endpoint could be a Server which has an asynchronous method accept() which yields a Future containing
% the Connection wrapper directly
% - Connection wrapper object could expose functions like accept_bi_stream() which again returns the stream object directly
% - this way the user has the advantage of a very intuitive API design and the library has the advantage of not having to
% manage these objects

The current design of the library keeps all connections within a vector inside the Endpoint struct. To manage these connections,
a hashmap is used in which the current connection ID acts as the key and the corresponding index (\inlinecode{Handle}) in the
vector serves as the value. This approach necessitates lookups for every received packet and every executed event, potentially
becoming a performance bottleneck as the server aims to scale to handle hundreds of concurrent connections.

To address this concern and leverage the benefits of concurrent programming, a possible solution would be to shift the responsibility
of managing individual connections to the user. This could be achieved by wrapping the \inlinecode{Connection} and \inlinecode{Endpoint}
structs in new structures that expose a direct API of both underlying objects to the user.

For example, an \inlinecode{Endpoint} wrapper could be implemented as a \inlinecode{Server} struct. This \inlinecode{Server} would offer
an asynchronous \inlinecode{accept()} method that would yield a \inlinecode{Future} containing the wrapped \inlinecode{Connection}
object directly. Similarly, the Connection wrapper could expose functions like \inlinecode{accept\_bi\_stream()}, which again returns
a wrapped \inlinecode{Stream} object directly.

\begin{codeblock}{main.rs (Concept)}{Rust}
    \begin{rustcode}
        // copied from s2n-quic: https://github.com/aws/s2n-quic/blob/main/examples/rustls-mtls/src/bin/quic_echo_server.rs
        while let Some(mut connection) = server.accept().await {
            // spawn a new task for the connection
            tokio::spawn(async move {

                while let Ok(Some(mut stream)) = connection.accept_bidirectional_stream().await {
                    // spawn a new task for the stream
                    tokio::spawn(async move {

                        // echo any data back to the stream
                        while let Ok(Some(data)) = stream.receive().await {
                            stream.send(data).await.expect("stream should be open");
                        }
                    });
                }
            });
        }
    \end{rustcode}
    \label{example_lib_redesign}
\end{codeblock}

This approach offers a two-fold advantage. Users would benefit from a more intuitive and user-friendly API design, while the library
itself wouldn't need to deal with the internal management of these objects, resulting in improved performance and scalability.
Additionally this design would remove the need for individual events, as every action can now be performed on the object itself.
