\chapter{Evaluation}

Throughout the process of writing this thesis, the QUIC library accumulated 1855 lines of code (as of the submit date).
All major goals that have been set in advance to this thesis have been achieved, except for one. The QUIC library is able to receive
QUIC packets, parse, decrypt, process, and encrypt header and payload data through all three packet number spaces, and construct
answer packets. The only exception is the full TLS 1.3 1-RTT handshake. Due to numerous challenges (\ref{challenges}) and the short
time frame the full handshake has narrowly not been achieved. Additionally the original objective to test the implementation with
\inlinecode{aioquic} had to be adjusted to instead use \inlinecode{quinn}. 

Using Rust as only programming language proved to be a valuable lesson. Rusts concepts of ownership, borrowing and lifetimes dictate a
special style of programming that prevents handling of raw pointers and hence massively complicates the manipulation of data through
references outside of the scope of the data the pointers are referring to. Keeping references in structs for example is notoriously
difficult due to Rusts lifetime annotations. While all these measures have resulted in not a single segmentation fault being
thrown for the whole development phase, it gravely impacted the library design. One of those being that the \inlinecode{recv()}
and \inlinecode{accept()} functions from the \inlinecode{Connection} object always return into the endpoint to minimise the
amount of references having to be kept. Additionally, Rusts effort to improve the readability of compiler error messages
greatly accelerated debugging.

While one might think that the limitations of Rust may limit performance, the
\inlinecode{quiche}\footnote{\url{https://github.com/cloudflare/quiche}} implementation by cloudflare, also developed in Rust,
can even outperform \inlinecode{picoquic}\footnote{\url{https://github.com/private-octopus/picoquic}}, an implementation in
pure C\footnote{\url{https://www.diva-portal.org/smash/get/diva2:1691838/FULLTEXT03}}.

\section{Execution}
Rust and Cargo should enable fully functionaly cross-platform compilation as long as \inlinecode{rustc} version \inlinecode{1.72.0}
or higher is used. To test with an external client, clone the external library \inlinecode{quinn} first.

\begin{terminalblock}
    \begin{textcode}
        [user@computer thesis]$ git clone https://github.com/quinn-rs/quinn
        ...
        [user@computer thesis]$ cd quinn
    \end{textcode}
\end{terminalblock}

Open a second terminal and navigate to the \inlinecode{project} folder inside the thesis directory. Start the server by
executing \inlinecode{cargo run}.

\begin{terminalblock}
    \begin{textcode}
        [user@computer thesis] ~/thesis/project $ cargo run
    \end{textcode}
\end{terminalblock}

Switch back to the first terminal and start the client connection by executing the following command.

\begin{terminalblock}
    \begin{textcode}
        [user@computer thesis] ~/quinn $ cargo run --example client https://127.0.0.1:34254/Cargo.toml
    \end{textcode}
\end{terminalblock}

The second terminal window now shows the librarys debug statements.

\begin{terminalblock}
    \begin{textcode}
        [user@computer thesis] ~/thesis/project $ cargo run
        Received 1200 bytes from 127.0.0.1:63844
        0x00 version: 0x0001 pn: 0x00000000 dcid: 0xe5d0924ff82d5b9a scid: 0x365428d847571dab token:[] length:1162
        ...
    \end{textcode}
\end{terminalblock}

\section{Challenges} \label{challenges}
The vast scope of QUIC inherently leads to high complexity. This is evident from the very beginning - 
the implementation overhead needed to just be able to build an answer to an initial packet is immense.
Throughout the whole implementation process a carefully designed library is crucial to avoid major refactoring and
rewrites later on.

Unfortunately, the newness of QUIC means there's a scarcity of widely available implementations, tutorials,
and explanations. This lack of resources makes it more challenging to understand the protocol itself and derive a design
for the QUIC library.

Most of the major setbacks encountered during the implementation process stemmed from the Rustls library. Despite its full support for 
TLS 1.3 and an API specifically designed for QUIC, a significant portion of its functionality is undocumented. This lack of documentation
extends over most of the API, in that the explanations of function parameters is often incomplete or missing and examples with
broader contexts are missing entirely.
For instance, when initializing the RustlsConnection, it was unclear that the third parameter (\inlinecode{params}) referred to
RFC section 18. Similarly, when calling \inlinecode{read\_hs()}, which takes a reference to data, it wasn't clear where the
pointer should start - at the beginning of the CRYPTO frame or the TLS message within it. Further testing revealed that
the function returned an error, but not one of the standardized TLS 1.3 errors. To identify the issue, one has to explicitly
poll for TLS alerts using the \inlinecode{alerts()} function which wasn't mentioned anywhere (Fig. \ref{error_handling_read_hs}).

\begin{codeblock}{lib.rs}{Rust}
    \begin{rustcode}
        match self.tls_session.read_hs(crypto_frame.data()) {
            Ok(()) => println!("Successfully read handshake data"),
            Err(err) => {
                eprintln!("Error reading crypto data: {}", err);
                eprintln!("{:?}", self.tls_session.alert().unwrap());
            }
        }
    \end{rustcode}
    \label{error_handling_read_hs}
\end{codeblock}

The biggest challenge, however, was related to the application layer protocol negotiation. This is an obligatory field, carried
in any TLS client hello, and its contents are arbitrary but a standardized list of protocol codes is maintained by IANA, because
when negotiating a protocol, for example http/3, the protocol code has to match on both endpoints. The initial test client, built
with \inlinecode{aioquic} in Python, did not include this field, leading to an error and a TLS alert. After switching the test
client to the example HTTP/3 client provided by the quinn library, the ALPN field was included, but the connection still failed.
The Rustls documentation made no mention of its inability to handle the standardized IANA list. After digging deep
within the library code, it became apparent that ALPN protocols have to be specified manually outside of the usual configuration.
This involved directly setting the field within a specific struct of the RustlsConnection class. 

\section{Future Improvements}

While the current QUIC library provides a solid foundation, it lacks several key features that need to be implemented first, in
order to facilitate complete QUIC connections as per specification. The library is designed with the whole protocol in mind and
should be easily expandable without requiring a major restructuring effort. That being said, there are still a number of areas
within the existing codebase that could benefit from optimization and improvement, even if the pure QUIC implementation is
feature complete at some point in the future.

\subsection{Asychronous Design}

The current design of the QUIC library operates within a single thread. However, a production grade implementation requires
concurrency. Rust offers two primary models for achieving concurrency: OS threads and asynchronous runtime programming.

Asynchronous programming excels in scenarios dominated by I/O bound tasks, which is precisely the case for network applications
like servers and databases. It significantly reduces CPU and memory overhead compared to traditional thread-based approaches.
An async runtime manages a limited pool of expensive OS threads more efficiently. These threads are optimised to handle a much larger
number of lightweight tasks, enabling significantly more concurrent operations compared to using raw OS threads directly.

Unfortunately, transitioning the current library to a concurrent model would necessitate a major rework, particularly for the
\inlinecode{Connection} and \inlinecode{Endpoint} components. Facilitating concurrent reading and writing of serveral resources 
would require significant changes. For instance, the connection "database" design might need to be guarded by a mutex or even
removed entirely in favor of a different approach (see Sec. \ref{connection_db_rework}).
Additionally, the \inlinecode{recv()} function for example would need to return a \inlinecode{Future} object, requiring
the use of await statements to retrieve data in an asynchronous environment, along with multiple other functions.

The transition to an asychronous design would have major performance benefits but would require a major effort as Rusts asynchronous
features are widely known to require signifiant effort to work with and are still
maturing\footnote{\url{https://bitbashing.io/async-rust.html}}.

\subsection{Connection Database Rework in Endpoint} \label{connection_db_rework}

The current design of the library keeps all connections within a vector inside the Endpoint struct. To manage these connections,
a hashmap is used in which the current connection ID acts as the key and the corresponding index (\inlinecode{Handle}) in the
vector serves as the value. This approach necessitates lookups for every received packet and every executed event, potentially
becoming a performance bottleneck as the server aims to scale to handle hundreds of concurrent connections.

To address this concern and leverage the benefits of concurrent programming, a possible solution would be to shift the responsibility
of managing individual connections to the user. This could be achieved by wrapping the \inlinecode{Connection} and \inlinecode{Endpoint}
structs in new structures that expose a direct API of both underlying objects to the user.

For example, an \inlinecode{Endpoint} wrapper could be implemented as a \inlinecode{Server} struct. This \inlinecode{Server} would offer
an asynchronous \inlinecode{accept()} method that would yield a \inlinecode{Future} containing the wrapped \inlinecode{Connection}
object directly. Similarly, the Connection wrapper could expose functions like \inlinecode{accept\_bi\_stream()}, which again returns
a wrapped \inlinecode{Stream} object directly.

\begin{codeblock}{main.rs (Concept)}{Rust}
    \begin{rustcode}
        // copied from s2n-quic: https://github.com/aws/s2n-quic/blob/main/examples/rustls-mtls/src/bin/quic_echo_server.rs
        while let Some(mut connection) = server.accept().await {
            // spawn a new task for the connection
            tokio::spawn(async move {

                while let Ok(Some(mut stream)) = connection.accept_bidirectional_stream().await {
                    // spawn a new task for the stream
                    tokio::spawn(async move {

                        // echo any data back to the stream
                        while let Ok(Some(data)) = stream.receive().await {
                            stream.send(data).await.expect("stream should be open");
                        }
                    });
                }
            });
        }
    \end{rustcode}
    \label{example_lib_redesign}
\end{codeblock}

This approach offers a two-fold advantage. Users would benefit from a more intuitive and user-friendly API design, while the library
itself wouldn't need to deal with the internal management of these objects, resulting in improved performance and scalability.
Additionally this design would remove the need for individual events, as every action can now be performed on the object itself.
