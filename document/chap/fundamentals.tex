\chapter{Fundamentals}

\section{Terms and Definitions}

\begin{itemize}
    \item \textbf{Endpoint:} An entity that can participate in a QUIC connection by generating, receiving, and processing QUIC packets. 
    There are only two types of endpoints in QUIC: client and server.
    \item \textbf{QUIC packet:} A complete processable unit of QUIC that can be encapsulated in a UDP datagram. One or more QUIC
    packets can be encapsulated in a single UDP datagram.
    \item \textbf{Variable Sized Integer} QUIC uses two most significant bits of the first byte to encode the base-2 logarithm of the
    integer encoding length in bytes. The remaining bits store the integer value itself, in network byte order. This allows encoding
    integers in 1, 2, 4, or 8 bytes, representing 6-, 14-, 30-, or 62-bit values. \label{variable_sized_integer}
    \item \textbf{Explicit Congestion Notification (ECN)}: ECN is an optional mechanism within the Internet Protocol (IP) that allows
    routers to signal incipient congestion to data senders before packet drops occur. This is achieved by routers setting a dedicated ECN
    bit in the IP header of packets traversing congested paths. \label{ecn}
    \item \textbf{ACK Range:} An ACK range specifies a continuous sequence of packets that have been successfully received by the receiver. When a
    receiver acknowledges data, it doesn't necessarily need to acknowledge each packet individually. Instead, it can efficiently transmit an ACK
    with a range, indicating that it has received all packets within that range, from a starting point to an ending point. \label{ack_range}
    \item \textbf{Connection ID:} An identifier that is used to identify a QUIC connection at an endpoint. Each endpoint selects one
    or more connection IDs for its peer to include in packets sent towards the endpoint. This value is opaque to the peer. \label{cid}
    \item \textbf{Authenticated Encryption with Associated Data (AEAD):} AEAD is a cryptographic mode that builds upon Authenticated Encryption (AE).
    AEAD allows messages to include Associated Data (AD) alongside the actual message to be encrypted. This associated
    data is additional information that is not confidential but requires authenticity and integrity protection. In essence, AEAD authenticates the
    confidentiality of the encrypted message and the integrity of both the message and the associated data. This makes AEAD suitable for scenarios
    like network packets, where the header containing routing information needs to be integrity-protected and authenticated, while the payload
    carrying the actual content is encrypted for confidentiality. \label{aead}
\end{itemize}

%\section{Fundamentals of networking protocols}

%\section{TCP \& UDP}