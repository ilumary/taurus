\chapter{Summary}
QUIC, officially released in 2021 as a successor to TCP, aims to address longstanding issues affecting its predecessor. TCP over TLS
suffers from problems such as Head-of-Line Blocking, lengthy handshakes, reliance on fixed IP addresses, and an inherent coupling
between congestion control and data reliability.

QUIC features a robust architecture designed to overcome these shortcomings. It allows for multiple concurrent uni-directional and
bi-directional streams within a single connection. Packets consist of headers and payloads, with payloads further divided into
frames. Each frame carries specific data based on the frames type. Headers come in two forms: a longer version for connection
establishment and a shorter version used during an established connection to minimize processing overhead. Notably, both the header
and the complete payload are individually protected for enhanced security.

Flow control within QUIC operates at both the stream and connection level, with adjustments communicated via dedicated frames. The
congestion control mechanism resembles TCP's approach, utilizing slow start and a cubic algorithm. Additionally, QUIC introduces
other features such as a 0-RTT (Zero Round Trip Time) handshake which allows for immediate application data transmission upon
re-establishing a recently turned inactive connection. Additionally, connection migration enables connections to seamlessly
switch between network interfaces without disruption. During a typical 1-RTT (One Round Trip Time) QUIC handshake, TLS integration
enables application data transfer after a single round trip.

Connections can be terminated due to timeouts, immediate closure, or stateless resets. QUIC employs acknowledgment (ACK) frames for
loss detection, with a single ACK frame capable of acknowledging multiple QUIC packets. Frames within a lost packet can be retransmitted
across multiple succeeding packets for improved reliability. The security of QUIC has been rigorously tested against various attack vectors,
including replay attacks, packet manipulation, and fuzzing. Interestingly, fuzzing proved to be the most effective method, causing
a server crash by manipulating offsets within stream frames.

A key element of this thesis is a dedicated QUIC implementation in Rust. Even though the implementation can not yet be considered
feature complete, it contains the most important core functionality. It features an event based design which proved highly useful
in conveying important information between the application and the connection.

The \inlinecode{Endpoint} object, handles a UDP socket and manages connections. Incoming packets are either matched
to an existing connection or trigger the creation of a new one. Each \inlinecode{Connection} object encapsulates the TLS
session provided by Rustls, manages received packets across all packet number spaces, handles keying material, and
populates outgoing packets with data. Encryption and decryption of packet headers and payloads are handled by functions
provided by Rustls.

The software design is highly influenced by Rusts characteristics and feature set, specifically the borrow checker, which
prevented memory related bugs but altered the design process. While some parts of the language
are highly complex, such as the lifetime annotations when working with references, the compiler always provided clear and
useful error messages resulting in only a small amount of the overall time used for bug fixing. Rust promotes to
minimise usage of mutable variables and references and encourages to use the stack in favor of the heap as much as possible.

Time constraints resulted in remaining potential for further improvements. These include the adoption of
an asynchronous design and shifting the responsibility of connection management to the user to reduce workload
in the endpoint.