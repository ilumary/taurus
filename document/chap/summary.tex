\chapter{Summary}
QUIC, officially released in 2021 as a successor to TCP, aims to address longstanding issues plaguing its predecessor. TCP suffers
from problems such as Head-of-Line Blocking, lengthy handshakes, reliance on fixed IP addresses, and an intricate coupling between
congestion control and data reliability.
QUIC boasts a robust architecture designed to overcome these shortcomings. It allows for multiple concurrent uni-directional and
bi-directional streams within a single connection. Packets consist of headers and payloads, with payloads further divided into
frames. Each frame carries specific data based on the frames type type. Headers come in two forms: a longer version for connection
establishment and a shorter version used during an established connection to minimize processing overhead. Notably, both the header
and the complete payload are individually protected for enhanced security.
Flow control within QUIC operates at both the stream and connection level, with adjustments communicated via dedicated frames. The
congestion control mechanism resembles TCP's approach, utilizing slow start and a cubic algorithm. Additionally, QUIC introduces
other features such as a 0-RTT (Zero Round Trip Time) handshake which allows for immediate application data transmission upon
re-establishing a recently turned inactive connection. Additionally, connection migration enables connections to seamlessly
switch between network paths without disruption. During a typical 1-RTT (One Round Trip Time) QUIC handshake, TLS integration
enables application data transfer after a single round trip.
Connections can be terminated due to timeouts, immediate closure, or stateless resets. QUIC employs acknowledgment (ACK) frames for
loss detection, with a single ACK frame capable of acknowledging multiple QUIC packets. Lost frames within a packet can be retransmitted
across multiple packets for improved reliability. The security of QUIC has been rigorously tested against various attack vectors,
including replay attacks, packet manipulation, and fuzzing. Interestingly, fuzzing proved to be the most effective method, causing
a server crash by manipulating offsets within stream frames.

Due to time constraints, the implemented QUIC library focuses on a core subset of the protocol's features. This is further reflected
in the decision to leverage Rustls for a complete TLS implementation, instead of implementing it alongside QUIC. The library itself
is designed to accommodate the full QUIC protocol without requiring major restructuring efforts.
The core component, the \inlinecode{Endpoint} object, wraps a UDP socket and manages connections. Incoming packets are either matched
to an existing connection or triggers the creation of a new one. Each \inlinecode{Connection} object encapsulates the TLS session
provided by Rustls, manages received packets across all packet number spaces, handles keying material, and populates outgoing packets
with data. Encryption and decryption of packet headers and payloads are handled by functions provided by Rustls.
A packet is first parsed into a dedicated \inlinecode{Header} object. Depending on the frame size within the payload, additional parsing
occurs to create specific structs for each contained frame. Following this parsing step, the payload is processed. Each processed
packet generates at least one event that the user can act upon. Finally, the connection prepares a response packet containing
relevant information such as acknowledgments and outstanding cryptographic data.

While the library boasts a well-considered design, there is substantial potential for improvement. This includes the adoption of
an asynchronous design or shifting the responsibility of connection management to the user.